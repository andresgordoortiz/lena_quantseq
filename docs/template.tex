\documentclass[10pt,twocolumn]{article}

% ============================================================================
% PACKAGES
% ============================================================================
\usepackage[utf8]{inputenc}
\usepackage[T1]{fontenc}
% Traditional LaTeX fonts (Computer Modern) - no font package needed

\usepackage[margin=1cm]{geometry}
\usepackage{graphicx}
\graphicspath{{../}{./}}  % look in parent dir (project root) and current dir
\usepackage{float}
\usepackage{caption}
\usepackage{subcaption}
\usepackage{amsmath}
\usepackage{booktabs}
\usepackage{hyperref}
\usepackage{xcolor}
\usepackage{listings}


% ============================================================================
% TITLE INFORMATION
% ============================================================================
\title{Analysis Report on the Quant-Seq Data from Lena}
\author{Andrés Gordo\textsuperscript{1}  \\[0.5em]
    \small \textsuperscript{1}IMP, Vienna, Austria \\
}
\date{\today}

% ============================================================================
% STYLING
% ============================================================================
\hypersetup{
    colorlinks=true,
    linkcolor=blue!60!black,
    citecolor=green!50!black,
    urlcolor=blue!70!black
}

\setlength{\parindent}{1em}
\setlength{\parskip}{0.5em}

% Reduce spacing around sections
\usepackage{titlesec}
\titlespacing*{\section}{0pt}{0.5em}{0.3em}
\titlespacing*{\subsection}{0pt}{0.4em}{0.2em}

% Reduce spacing around floats
\setlength{\floatsep}{0.5em}
\setlength{\textfloatsep}{0.5em}
\setlength{\intextsep}{0.5em}
\setlength{\abovecaptionskip}{0.3em}
\setlength{\belowcaptionskip}{0.2em}

% Reduce spacing around listings
\lstset{
    aboveskip=1em,
    belowskip=1em
}

% ============================================================================
% DOCUMENT
% ============================================================================
\begin{document}

\maketitle

% ----------------------------------------------------------------------------
\section{Preprocessing \& Parameters}
% ----------------------------------------------------------------------------
Following our meeting, I decided to re-run the upstream processing of the Quant-Seq samples using the latest build of the \textit{Danio rerio} genome (GRCz12tu; Apr 4, 2025). I tried to keep the analysis consistent with what Maria did previously, running a UMI-sensitive pipeline with the following command:

\begin{lstlisting}[
    basicstyle=\footnotesize\ttfamily,
    breaklines=true,
    breakatwhitespace=true,
    columns=flexible,
    keepspaces=true
]
cd $SCRATCH/lena_quantseq
sbatch submit_nextflow_clip.sh \
  nf-core/rnaseq \
  --input samplesheet.csv \
  --outdir results/ \
  --fasta reference/genome.fna \
  --gtf reference/genome.gtf \
  --with_umi \
  --umitools_extract_method regex \
  --umitools_bc_pattern \
    "'^(?P<cell_1>.{6})(?P<umi_1>.{4}).*'" \
  --featurecounts_group_type gene_id \
  -c nextflow.config \
  -with-tower
\end{lstlisting}

The raw output files for the pipeline can be found in \verb|/bioinfo/andres.gordo/rnaseq_nfcore_out_umi.tar.gz|. The main output file that was used for downstream analysis is the gene counts matrix \verb|salmon.merged.gene_counts.tsv|. Normalisation and differential expression analysis was performed using DESeq2 in R (see code for details), with log2 fold-change shrinkage applied via a zero-centred Normal prior (\texttt{lfcShrink}, \texttt{type = "normal"}) to stabilise extreme effect-size estimates. From now onwards, a differentially expressed gene (DEG) is defined as one with an adjusted p-value $< 0.05$ (BH) and an absolute log2 fold change $\geq 1.0$. Genes with less than 10 counts across at least 3 samples were filtered out prior to DE analysis. This, as will be observed later, resulted in wnt11---member of the Nodal score---being lost.

\begin{figure}[H]
\centering
\includegraphics[width=\columnwidth]{results/compare_annotations/annotation_comparison_figure.pdf}
\caption{Annotations differences between the GRCz11 and GRCz12tu builds, showing the number of genes that were added, removed, or kept in each biotype category. None of the DEGs obtained in the following analyses were affected by these changes.}
\label{fig:annotation_changes}
\end{figure}

I also quickly compared the differences between the GRCz11 and GRCz12tu builds, and found that the biggest difference lies in an increase in the number of lncRNAs, which are not relevant for this analysis. I also observed some protein-coding additions and removals (mostly computational), although none of the DEGs obtained in the following analyses were affected by those changes (Fig~\ref{fig:annotation_changes}).

\section{Maria's Re-Analysis}

I started by quickly re-exploring one of Maria's previous analyses to 1) ensure that I could reproduce her results and 2) investigate if any newly-annotated gene popped up as DEG. Figure~\ref{fig:single} shows a result from Experiment 1, whereby the exposure time, but not the concentration, of Activin led to significant changes in the expression of genes from the Nodal score.

% --- Single column figure ---
\begin{figure}[H]
\centering
\includegraphics[width=\columnwidth]{results/nodal_heatmap_exp1_averaged.pdf}
\caption{Z Score of counts from the Nodal score genes across different time points and Activin concentration in Experiment 1.}
\label{fig:single}
\end{figure}




\begin{figure}[H]
\centering
\includegraphics[width=\columnwidth]{results/pca_combined.pdf}
\caption{PCA of gene expression across different time points and Activin concentration in Experiment 1 and Experiment 2.}
\label{fig:pca_combined}
\end{figure}

\subsection{Outlier identification}
% TODO: Write about the outlier S343239 (SB50 60min, Exp2)
% Visible on PCA; confirmed by near-zero Nodal score.

\begin{figure}[H]
\centering
\includegraphics[width=\columnwidth]{results/q2_nodal_score_cumulative_with_outlier.pdf}
\caption{Nodal score cumulative expression with all samples including the identified outlier (S343239, SB50 60\,min). The outlier is labelled and shows near-zero Nodal pathway activity.}
\label{fig:nodalscore_outlier}
\end{figure}

\section{Consistency of Activin Response}

\begin{figure}[H]
\centering
\includegraphics[width=\columnwidth]{results/q1_combined_volcano.pdf}
\caption{Volcano plots of DEGs in each experiment. In purple the shared genes. DEGs defined as adjusted p-value $<= 0.05$ and absolute log2 fold change $>= 1.0$.}
\label{fig:exp1_vs_exp2}
\end{figure}

The first thing on the list was to compare the DGE expression for $0$~ng/ml Activin vs.\ $15$~ng/ml Activin in Experiment~1 vs.\ Experiment~2. As can be observed in Figure~\ref{fig:exp1_vs_exp2}, there is good consistency between both experiments, with most genes following the same trend (up- or down-regulated). Experiment~1 yielded $1{,}311$ DEGs, while Experiment~2 had $1{,}076$ DEGs, with a total of $845$ DEGs in common. A quick look at one of the top genes, id3, reveals it as a DNA remodeller that is expressed at the dorsal side of the embryo. The list of all and shared DEGs can be found in q1\_exp1\_0vs15\_activin\_240min.csv, q1\_exp2\_0vs15\_dmso.csv, and q1\_shared\_de\_genes.csv, respectively.

\section{Nodal Score Exploration}

Next on the list was to re-explore how the selected list of genes belonging to the Nodal score varied in expression after 240 minutes of constant Activin exposure or Activin inhibition by SB50 at different timepoints. Figure~\ref{fig:nodalscore} shows the cumulative read counts of those Nodal score genes depending on the treatment. The `clean' treatment with Activin reaches the highest stimulation, whereas treatment with SB50 dampens the response. As was observed before, the longer the initial exposure to Activin, the more irreversibly committed the cells are, thereby responding less to the SB50 treatment. The statistical test resembles the one performed for GSEA (see code for details).

\begin{figure}[H]
\centering
\includegraphics[width=\columnwidth]{results/q2_nodal_score_cumulative.pdf}
\caption{Nodal score cumulative expression across treatments. A Kruskal-Wallis test was performed to assess significance across all conditions, followed by pairwise Wilcoxon tests with BH correction. Asterisks indicate significant pairwise comparisons against the control (0ngml DMSO): * p $< 0.05$, ** p $< 0.01$, *** p $< 0.001$.}
\label{fig:nodalscore}
\end{figure}

\subsection{Nodal score divergence drivers}
% TODO: Write about which genes drive the divergence between the Activin and
% SB50-180min lines at the endpoint (Figure~\ref{fig:nodal_divergence}).

\begin{figure}[H]
\centering
\includegraphics[width=\columnwidth]{results/q2_nodal_divergence_drivers.pdf}
\caption{Genes driving the divergence between the 15\,ng/ml Activin and SB50 180\,min conditions at endpoint, ranked by difference in cumulative normalised counts.}
\label{fig:nodal_divergence}
\end{figure}

\begin{figure}[H]
\centering
\includegraphics[width=\columnwidth]{results/q2_nodal_divergence_gene_panels.pdf}
\caption{Per-gene expression across conditions for the top divergence-driving Nodal score genes.}
\label{fig:nodal_divergence_panels}
\end{figure}


\section{Blocking Effect of SB50}

Next, I wanted to explore how many of the genes that were altered by Activin were effectively blocked by the SB50 treatment. In Experiment~2, SB50 severs activin signaling at the time of administration (60, 120, or 180\,min), and all samples are collected at 240\,min. Thus, a gene classified as ``Blocked'' at 60\,min means that 60\,min of activin exposure was \emph{not} sufficient for the gene to commit to the activin-induced expression program---once activin was cut off, the gene returned to baseline by collection time. Conversely, ``Not blocked'' means the gene committed within that exposure window: even though activin was severed, its effect persisted at 240\,min.

I found a total of 826 DEGs comparing Activin vs.\ the baseline control in Experiment~2. Of these, 533 were upregulated and 293 were downregulated. On the other hand, 227, 609, and 887 DEGs were found after severing activin with SB50 at 60, 120, and 180\,min, respectively. This is consistent with the idea that longer activin exposure leads to more irreversible commitment. Figure~\ref{fig:sb50_blocking} shows that the highest proportion of blocked (uncommitted) genes occurs at 60\,min, when cells had the shortest activin exposure. As the exposure window increases, more genes commit and become unblockable. Importantly, no gene shows a reversed response: genes are either blocked (uncommitted), not blocked (committed), or SB50-specific (activin-independent SB50 effects).

\begin{figure}[H]
\centering
\includegraphics[width=\columnwidth]{results/q3_analysis1_blocking.pdf}
\caption{SB50 blocking of Activin-induced genes. Scatter plots show Activin effect (x-axis) vs.\ SB50+Activin effect (y-axis) for each activin exposure window. Categories: \textcolor[HTML]{2166AC}{\textbf{Blocked}} (uncommitted: Activin-DE but reversed by SB50 cutoff), \textcolor[HTML]{B2182B}{\textbf{Not blocked}} (committed: DE persists despite SB50), \textcolor[HTML]{1B7837}{\textbf{SB50-specific}} (activin-independent SB50 effect). Bar chart shows how longer activin exposure leads to more committed genes.}
\label{fig:sb50_blocking}
\end{figure}

\subsection{Robustness of blocking classification}
\label{sec:blocking_robustness}

The main blocking figure above includes all four gene categories. To evaluate the robustness of the blocking classification, I repeated the analysis with two modifications (Figure~\ref{fig:blocking_variants}). First, I removed the SB50-specific genes (panel~a), since these are not Activin-responsive and may obscure the blocking proportions. This focuses the bar chart exclusively on the question: of genes altered by Activin, what fraction does SB50 block? Second, I restricted the analysis to genes that are DEG in both Experiment~1 and Experiment~2 (panel~b). Using only shared DEGs provides a more stringent gene set and tests whether the blocking pattern is reproducible.

\begin{figure}[H]
\centering
% Panel a: No SB50-specific
\includegraphics[width=\columnwidth]{results/q3_blocking_no_sb50specific.pdf}
\smallskip

\includegraphics[width=\columnwidth]{results/q3_blocking_shared_degs.pdf}
\caption{Robustness of SB50 blocking. \textbf{(Top)} Analysis restricted to Activin-responsive genes only (SB50-specific removed). \textbf{(Bottom)} Analysis restricted to DEGs shared between Exp~1 and Exp~2. In both cases, the blocking gradient across timepoints is preserved.}
\label{fig:blocking_variants}
\end{figure}

\subsection{Gene category transitions across timepoints}
\label{sec:gene_transfer}

An important question is what happens to individual genes as activin exposure time increases: do genes that are uncommitted (blocked) at 60\,min remain uncommitted, or do they commit as the exposure window grows? Figure~\ref{fig:gene_transfer} tracks these transitions. Each bar shows, for a given starting status, how many genes transition to each destination as activin exposure increases by one interval. The blue bars indicate genes whose activin response remains reversible (still uncommitted), while the red bars show genes that have committed. ``Newly committed'' (orange) are genes that cross the commitment threshold during that interval---these define the critical exposure window for each gene set. To understand which biological functions characterise each transition, I ran GO enrichment (Biological Process and Molecular Function) on the same pairwise-transition gene sets that correspond to each bar (Figure~\ref{fig:gene_transfer_go}).

\begin{figure}[H]
\centering
\includegraphics[width=\columnwidth]{results/q3_gene_transfer_plot.pdf}
\caption{Gene commitment transitions between activin exposure windows. Each bar shows how genes with a given status redistribute as activin exposure increases. Blue: still uncommitted; orange: newly committed; red: already committed. Left: 60\,$\to$\,120\,min; right: 120\,$\to$\,180\,min.}
\label{fig:gene_transfer}
\end{figure}

\begin{figure*}[t]
\centering
\includegraphics[width=\textwidth]{results/q3_gene_transfer_go.pdf}
\caption{Gene commitment dynamics and GO enrichment (Biological Process + Molecular Function) for each pairwise commitment transition. The upper panel shows the barplot of gene transitions between consecutive SB50 time-points (60$\to$120\,min and 120$\to$180\,min). The lower panel shows GO enrichment for the exact gene sets that form each bar, so gene counts in the facet labels correspond directly to the bar heights above. This reveals which biological functions are enriched among genes that remain uncommitted, newly commit, are already committed, or revert their commitment at each interval.}
\label{fig:gene_transfer_go}
\end{figure*}

\subsection{Expression patterns of blocked genes}
% TODO: Write about the expression patterns of genes blocked by SB50.

\begin{figure}[H]
\centering
\includegraphics[width=\columnwidth]{results/q3_blocked_genes_expression.pdf}
\caption{Expression heatmap (z-scored) of top 30 genes classified as ``Blocked'' at the 60\,min timepoint, across all conditions.}
\label{fig:blocked_expression}
\end{figure}

\subsection{GO enrichment of SB50-specific genes}
\label{sec:sb50_specific_go}

The SB50-specific (green) category is biologically interesting: these genes are \emph{not} altered by Activin alone, but become significantly differentially expressed only when SB50 is added. They represent activin-independent effects of SB50---pathways that SB50 activates or represses regardless of TGF-$\beta$ signalling. Figure~\ref{fig:sb50_specific_go} shows GO enrichment across Biological Process, Molecular Function, and Cellular Component ontologies, faceted by activin exposure time. Since the enriched terms differ substantially between timepoints (reflecting distinct gene sets at each window), each row shows its own top terms rather than forcing a shared axis.

\begin{figure*}[t]
\centering
\includegraphics[width=\textwidth]{results/q3_sb50_specific_go.pdf}
\caption{GO enrichment of SB50-specific genes (green category) faceted by activin exposure time (rows) and ontology (columns). Each panel shows the top enriched terms for that specific timepoint and ontology. These activin-independent SB50 effects vary across exposure times, likely reflecting different cellular states at the time of SB50 administration.}
\label{fig:sb50_specific_go}
\end{figure*}


\section{SB50 Effect vs.\ Timed Exp 1}

A similar yet different analysis from the previous is the question of comparing the effect of SB50 treatment with the corresponding matching times of collection in Experiment~1. Figure~\ref{fig:sb50_vs_time} shows this approach. The way I interpret it is manifold. First, looking at the X axis, which compares the effect of SB50 at endpoint with the effect of Activin at endpoint (both from Experiment~2), we again observe how SB50 is more effective at blocking---downregulating---genes when added at the earliest point (orange points more abundant on the left side; they decrease in number as we increase the treatment time). Second, looking at the Y axis, we are comparing instead the effect at endpoint of SB50 (Experiment~2) with the state of cells that were exposed to Activin at a given timepoint. Therefore, the gene expression differences between these two conditions reflect two different temporal states. In the first plot (top left), we compare adding SB50 at 60min and collecting at 240 minutes with Activin-induced cells collected at 60 minutes. As expected, many genes are differentially expressed (mostly upregulated) between 60 minutes and 240 minutes, regardless of the SB50 treatment. However, as we delay not only the time of SB50 addition but also of Activin-induced cells collection, the two conditions temporally converge, and the number of DEGs decreases.

These two figures suggest that:
\begin{enumerate}
  \item SB50 is more effective at blocking Activin-induced genes the earlier it is added, hinting at a cell commitment process that is time-dependent.
  \item Uncoupling temporal states from Activin inductions allows us to dissect which genes are Activin-dependent vs.\ time-dependent.
\end{enumerate}

\begin{figure}[H]
\centering
\includegraphics[width=\columnwidth]{results/q3_analysis2_comparisons.pdf}
\caption{Comparison of gene expression changes induced by SB50 treatment at endpoint vs.\ activin-induced changes at different timepoints. Each panel corresponds to a different SB50 addition time (60min, 120min, 180min).}
\label{fig:sb50_vs_time}
\end{figure}


\section{Gene Candidates \& ChIP-Seq}

Finally, I explored the dynamics of gene expression of the list of candidates genes alongside the ones encompasing the nodal Score, and integrated insights from the publsihed 2014 ChIP-Seq dataset of Smad2 and Eomesa. To do that, I first asked whether these genes appeared in either ChIP-seq dataset. Figure~\ref{fig:chip} shows that information, including if the FoxH1 motif was also found around the peak.


\begin{figure}[t]
\centering
\includegraphics[width=\columnwidth]{results/q4_chipseq_binding_heatmap.pdf}
\caption{Heatmap showing if the candidate or nodal score genes were found in the ChIP-seq experiment, including the presence of the FoxH1 motif around the peak.}
\label{fig:chip}
\end{figure}

Next, I analysed the ppaterns of gene expression in respone to activin (Exp 1) of these candiadte genes and the Nodal score. I found that, except sna1b--which is downregulated--, all genes are either actiavted consistently, or activated at early timepoints and then downregulated at later timepoints. Thisis shown in Figure~\ref{fig:temporal_clusters}.

\begin{figure*}[t]
\centering
\includegraphics[width=\textwidth]{results/q4_temporal_clusters.pdf}
\caption{Heatmap showing the temporal patterns of gene expression in response to Activin (Experiment 1) for the candidate genes and the Nodal score.}
\label{fig:temporal_clusters}
\end{figure*}

Consequently, Figure~\ref{fig:reversibility_profile} shows the same genes but how they respond to the Experiment 2 treatments.

\begin{figure}[t]
\centering
\includegraphics[width=\columnwidth]{results/q4_reversibility_profile.pdf}
\caption{Heatmap showing the temporal patterns of gene expression in response to Activin (Experiment 2) for the candidate genes and the Nodal score.}
\label{fig:reversibility_profile}
\end{figure}

And finally, Figure~\ref{fig:integrated_overview} shows an integrated overview of gene expression patterns for the candidate genes and the Nodal score across both experiments.


\begin{figure}[t]
\centering
\includegraphics[width=\columnwidth]{results/q4_integrated_overview.pdf}
\caption{Heatmap showing the integrated overview of gene expression patterns for the candidate genes and the Nodal score across both experiments.}
\label{fig:integrated_overview}
\end{figure}

\subsection{ChIP-seq integration}
% Two focused heatmaps: one for Nodal pathway + candidate genes, one for DUSP + FGF pathway genes.

\begin{figure}[t]
\centering
\includegraphics[width=\columnwidth]{results/q4_chipseq_nodal_candidates.pdf}
\caption{ChIP-seq binding evidence for Nodal score genes and candidate genes. Filled tiles indicate Smad2 or EomesA binding peaks; Foxh1 motif presence is shown separately.}
\label{fig:chip_nodal_candidates}
\end{figure}

\begin{figure}[t]
\centering
\includegraphics[width=\columnwidth]{results/q4_chipseq_dusp_fgf.pdf}
\caption{ChIP-seq binding evidence for the DUSP phosphatase family, FGF direct targets, FGF ligands, and FGF receptors. This combined view replaces the earlier separate FGF and DUSP heatmaps.}
\label{fig:chip_dusp_fgf}
\end{figure}

\subsection{Gene family temporal dynamics}
% TODO: Write about temporal expression patterns of FGF ligands, DUSP family,
% mesoderm/FGF-responsive genes, and FGF receptors in response to Activin.

\begin{figure*}[t]
\centering
\includegraphics[width=\textwidth]{results/q4_gene_families_temporal.pdf}
\caption{Temporal expression dynamics of gene families (FGF targets, FGF ligands, DUSP phosphatases, mesoderm/FGF-responsive genes, FGF receptors) in response to 15\,ng/ml Activin in Experiment~1. Thin lines show individual genes; thick lines show family means.}
\label{fig:gene_families}
\end{figure*}

\subsection{Integrated overview at 120\,min}
% TODO: Write about repeating the integrated view (peak timing vs
% reversibility) at 120 minutes of SB50 addition, in addition to 60 minutes.

\begin{figure}[t]
\centering
\includegraphics[width=\columnwidth]{results/q4_integrated_overview_120min.pdf}
\caption{Integrated overview of peak response timing vs.\ reversibility at SB50 addition after 120\,minutes, complementing the 60-minute analysis.}
\label{fig:integrated_120}
\end{figure}

\subsection{Cell migration DEGs}

\begin{figure*}[t]
\centering
\includegraphics[width=\textwidth]{results/q4_migration_degs_temporal.pdf}
\caption{Temporal expression of shared DEGs annotated to cell migration GO terms (GO:0016477, GO:0030334, GO:0040011) during Activin response, separated into upregulated (left) and downregulated (right) panels.}
\label{fig:migration}
\end{figure*}

\subsection{Cell adhesion DEGs}

\begin{figure*}[t]
\centering
\includegraphics[width=\textwidth]{results/q4_adhesion_degs_temporal.pdf}
\caption{Temporal expression of shared DEGs annotated to cell adhesion (GO:0007155), excluding genes already shown in the migration panel. Upregulated (left) and downregulated (right) genes are shown separately.}
\label{fig:adhesion}
\end{figure*}

% FGF signaling ChIP-seq subsection removed — content is now covered by
% the combined DUSP+FGF heatmap in Figure~\ref{fig:chip_dusp_fgf}.

\section{Molecular Function}

Next, I performed a quick Gene Ontology enrichment analysis using the clusterProfiler package in R. I focused on the Molecular Function ontology and explored the top enriched terms for the different comparisons performed so far. Figure~\ref{fig:go_mf} shows a selection of interesting terms that were enriched in different comparisons. The Exp~1 figure shows the Gene Ontology terms of the DEGs of each Activin concentration at each timepoint compared to the non-induced control samples. It shows predominancy of terms related to growth, signalling pathways, and transcription. Interestingly, the terms related to signalling receptors appear in every condition except for the latest timepoint, regardless of concentration, suggesting a negative feedback loop whereby Activin at first prompts the transcription of signalling receptors, but later on these get downregulated again.



\begin{figure*}[t]
\centering
\includegraphics[width=\textwidth]{results/go_mf_combined_publication.pdf}
\caption{Molecular Function GO term enrichment analysis for Experiment 1 (top) and Experiment 2 (bottom). In Experiment 1, each panel corresponds to a different timepoint (60min, 120min, 240min), with different Activin concentrations (0ng/ml, 5ng/ml, 15ng/ml) compared to the control (0ng/ml). In Experiment 2, each panel corresponds to a different comparison: Activin vs.\ Control (left), SB50 vs.\ Control (centre), SB50 vs.\ Activin (right). Only the top 10 enriched terms per condition are shown.}
\label{fig:go_mf}
\end{figure*}

\subsection{Comparative GO: Exp~2 with Exp~1 terms}
% TODO: Write about showing the same top-15 Exp1 GO MF terms in Exp2
% comparisons, for a direct visual comparison of pathway alterations.

\begin{figure}[H]
\centering
\includegraphics[width=\columnwidth]{results/go_exp2_comparative_exp1_terms.pdf}
\caption{Exp~2 GO Molecular Function enrichment restricted to the top-15 terms identified in Exp~1, enabling direct comparison of pathway conservation across experiments.}
\label{fig:go_comparative}
\end{figure}

\subsection{GO: Motility and cell adhesion}
% TODO: Write about GO BP terms related to cell motility, migration, and
% adhesion across conditions.

\begin{figure}[H]
\centering
\includegraphics[width=\columnwidth]{results/go_motility_adhesion_bp.pdf}
\caption{GO Biological Process terms related to cell motility, migration, adhesion, and locomotion across Activin and SB50 conditions.}
\label{fig:go_motility}
\end{figure}

\subsection{GO: SB50-specific genes}
% TODO: Write about GO enrichment of genes exclusively regulated by SB50
% (the green/SB50-specific category from the blocking analysis).

\begin{figure}[H]
\centering
\includegraphics[width=\columnwidth]{results/go_sb50_specific_bp.pdf}
\caption{GO Biological Process enrichment for SB50-specific genes (not present as Activin-induced DEGs) at each SB50 timepoint.}
\label{fig:go_sb50_specific}
\end{figure}


\section{FGF--Nodal Gene Regulatory Network}

To investigate the regulatory crosstalk between Nodal/Activin and FGF signalling, I built a data-driven gene regulatory network (GRN) combining literature-curated edges with temporal correlation analyses from Experiment~1. Because exposure time---not Activin concentration---is the primary driver of transcriptomic changes in this dataset, all analyses focus on the 15\,ng/ml time course (15--240\,min vs.\ 0\,ng/ml controls). The analysis is motivated by the observation that Nodal activates FGF ligand transcription, yet prolonged Nodal exposure leads to FGF/ERK suppression via DUSPs, particularly Dusp4 (van Boxtel et~al.\ 2018), creating an \emph{incoherent feedforward loop}.

\subsection{Positive controls: literature-validated pathway genes}

To validate the temporal profiling framework, I first examined genes with well-established roles in the FGF--Nodal network (Figure~\ref{fig:grn_controls}). Nodal pathway genes (ndr1, ndr2, lefty1/2) activate earliest (15--30\,min), followed by Nodal-induced FGF ligands (fgf3, fgf17, fgf8a). FGF/MAPK feedback targets (etv4, etv5a, spry2, dusp6) respond next, confirming that Activin activates FGF signalling in this time course. Mesoderm genes (tbxta, tbx16) follow as expected for FGF-dependent targets. This sequential activation validates the analytical framework for candidate gene discovery.

\begin{figure*}[t]
\centering
\includegraphics[width=\textwidth]{results/grn_positive_controls.pdf}
\caption{Temporal expression dynamics of literature-validated pathway genes (log$_2$FC, 15 vs.\ 0\,ng/ml Activin). Thin lines: individual genes; thick lines: group means. The recovery of the expected activation sequence (Nodal $\to$ FGF ligands $\to$ FGF feedback $\to$ mesoderm) validates the temporal profiling approach.}
\label{fig:grn_controls}
\end{figure*}

\subsection{Candidate pathway modules}

Applying the identical framework to candidate gene groups reveals their temporal behaviour relative to the validated positive controls (Figure~\ref{fig:grn_candidates_temporal}). Of particular interest is the DUSP family: dusp6 responds immediately (consistent with its role as a direct FGF target), while dusp4 shows delayed activation---matching its proposed role as a Nodal-induced, rather than FGF-induced, negative regulator of ERK signalling. The Sprouty/Spred, BMP, and Wnt crosstalk modules provide additional context for pathway interactions.

\begin{figure*}[t]
\centering
\includegraphics[width=\textwidth]{results/grn_candidate_dynamics.pdf}
\caption{Temporal dynamics of candidate and survey gene modules, plotted identically to the positive controls for direct comparison. The DUSP family shows heterogeneous timing: dusp6 is an immediate FGF target, whereas dusp4 peaks later, consistent with Nodal-driven induction.}
\label{fig:grn_candidates_temporal}
\end{figure*}

\subsection{Activation timeline}

To visualise the sequential activation of GRN components, Figure~\ref{fig:grn_timeline} shows the onset (first timepoint with $|\text{log}_2\text{FC}| > 0.5$) and peak response time for selected genes, ordered chronologically. The timeline reveals the temporal cascade: Nodal ligands $\to$ FGF ligands $\to$ FGF/MAPK feedback $\to$ DUSPs $\to$ mesoderm/endoderm effectors.

\begin{figure}[H]
\centering
\includegraphics[width=\columnwidth]{results/grn_activation_timeline.pdf}
\caption{Activation timeline: onset (circle) to peak (diamond) for selected GRN genes, ordered by onset time. Colour indicates pathway membership. The delayed onset of dusp4 relative to direct FGF targets is consistent with its proposed role as a Nodal-induced brake on FGF/ERK.}
\label{fig:grn_timeline}
\end{figure}

\subsection{Temporal correlation structure}

I computed pairwise Pearson correlations of temporal profiles (at 15\,ng/ml) across GRN genes. Genes anti-correlated with FGF/MAPK feedback targets, yet co-expressed with Nodal pathway genes, are strong candidates for mediating Nodal-driven FGF inhibition. Figure~\ref{fig:grn_cor_heatmap} shows the correlation structure among selected genes spanning multiple pathway modules.

\begin{figure}[H]
\centering
\includegraphics[width=\columnwidth]{results/grn_correlation_heatmap.pdf}
\caption{Temporal profile correlation matrix for selected FGF--Nodal GRN genes. Blue: positive correlation (co-activation); red: anti-correlation. The clustering reveals pathway modules and identifies candidate genes whose temporal profiles oppose FGF readouts.}
\label{fig:grn_cor_heatmap}
\end{figure}

\subsection{Candidate negative feedback scoring}

Each gene received a composite score integrating five evidence streams: (i) anti-correlation with FGF feedback targets (FDR-significant only), (ii) positive correlation with Nodal pathway (FDR-significant only), (iii) delayed onset relative to FGF targets, (iv) Smad2/EomesA ChIP-seq binding (Nelson et~al.\ 2014), and (v) late upregulation pattern. Figure~\ref{fig:grn_candidates_score} shows the component-level breakdown for the top 20 candidates, making the scoring transparent and self-explanatory.

\begin{figure*}[t]
\centering
\includegraphics[width=\textwidth]{results/grn_candidate_scoring.pdf}
\caption{Top 20 candidate negative feedback regulators with scoring component breakdown. Left tile heatmap: individual component scores (0--1 scale) for anti-FGF correlation, pro-Nodal correlation, delayed onset, ChIP-seq binding, and late upregulation. Right bar chart: composite score (equally-weighted mean of all five components). Correlations are gated by Benjamini--Hochberg FDR $< 0.05$.}
\label{fig:grn_candidates_score}
\end{figure*}

\subsection{Gene regulatory network}

Figure~\ref{fig:grn_network} presents the decluttered core GRN ($\sim$28 genes) combining literature-curated regulatory edges with data-driven correlations ($|r| > 0.85$, FDR $< 0.05$). Critically, literature edges are visually distinguished by whether our expression data validates them: bold edges indicate cases where the observed temporal correlation matches the expected regulatory direction (activation $\rightarrow$ positive $r$; inhibition $\rightarrow$ negative $r$) at FDR $< 0.05$.

\begin{figure*}[t]
\centering
\includegraphics[width=\textwidth]{results/grn_network_graph.pdf}
\caption{FGF--Nodal core gene regulatory network. Nodes are coloured by pathway group. Bold solid edges: literature relationships validated by expression data (FDR $< 0.05$, matching sign). Thin solid edges: literature relationships not confirmed. Dashed edges: data-driven temporal co-/anti-expression ($|r| > 0.85$, FDR $< 0.05$). Blue = activation/co-expression; red = inhibition/anti-expression.}
\label{fig:grn_network}
\end{figure*}

\section{Conclusion}

So far I think this is one of the best bulk rnaseq datasets I have ever worked with in terms of quality: showing a high consistency in the transcriptomic response to Activin, specifically noting that exposure time is a more significant driver of gene expression changes than concentration. The analysis of the Nodal score and SB50 inhibitor treatments suggests a time-dependent cell commitment process: early inhibition effectively blocks Activin-induced genes, but later inhibition fails to reverse the phenotype as cells become "committed" to their fate. Finally, Gene Ontology enrichment points toward a negative feedback loop in signaling receptors, which are initially upregulated by Activin but subsequently downregulated, a trend that SB50 can only partially rescue depending on the timing of the intervention.

% --- Full width figure (spans both columns) ---
% \begin{figure*}[t]
%     \centering
%     \includegraphics[width=\textwidth]{path/to/image.pdf}
%     \caption{Caption for full-width figure spanning both columns.}
%     \label{fig:fullwidth}
% \end{figure*}

% --- Two subfigures side by side (full width) ---
% \begin{figure*}[t]
%     \centering
%     \begin{subfigure}[t]{0.48\textwidth}
%         \centering
%         \includegraphics[width=\textwidth]{path/to/imageA.pdf}
%         \caption{Panel A}
%         \label{fig:panelA}
%     \end{subfigure}
%     \hfill
%     \begin{subfigure}[t]{0.48\textwidth}
%         \centering
%         \includegraphics[width=\textwidth]{path/to/imageB.pdf}
%         \caption{Panel B}
%         \label{fig:panelB}
%     \end{subfigure}
%     \caption{Overall caption for both panels.}
%     \label{fig:panels}
% \end{figure*}

% ============================================================================

\end{document}
