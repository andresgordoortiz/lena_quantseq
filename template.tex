\documentclass[10pt,twocolumn]{article}

% ============================================================================
% PACKAGES
% ============================================================================
\usepackage[utf8]{inputenc}
\usepackage[T1]{fontenc}
% Traditional LaTeX fonts (Computer Modern) - no font package needed

\usepackage[margin=1cm]{geometry}
\usepackage{graphicx}
\usepackage{float}
\usepackage{caption}
\usepackage{subcaption}
\usepackage{amsmath}
\usepackage{booktabs}
\usepackage{hyperref}
\usepackage{xcolor}
\usepackage{listings}

% ============================================================================
% TITLE INFORMATION
% ============================================================================
\title{Analysis Report on the Quant-Seq Data from Lena}
\author{Andrés Gordo\textsuperscript{1}  \\[0.5em]
    \small \textsuperscript{1}IMP, Vienna, Austria \\
}
\date{\today}

% ============================================================================
% STYLING
% ============================================================================
\hypersetup{
    colorlinks=true,
    linkcolor=blue!60!black,
    citecolor=green!50!black,
    urlcolor=blue!70!black
}

\setlength{\parindent}{1em}
\setlength{\parskip}{0.5em}

% Reduce spacing around sections
\usepackage{titlesec}
\titlespacing*{\section}{0pt}{0.5em}{0.3em}
\titlespacing*{\subsection}{0pt}{0.4em}{0.2em}

% Reduce spacing around floats
\setlength{\floatsep}{0.5em}
\setlength{\textfloatsep}{0.5em}
\setlength{\intextsep}{0.5em}
\setlength{\abovecaptionskip}{0.3em}
\setlength{\belowcaptionskip}{0.2em}

% Reduce spacing around listings
\lstset{
    aboveskip=1em,
    belowskip=1em
}

% ============================================================================
% DOCUMENT
% ============================================================================
\begin{document}

\maketitle

% ----------------------------------------------------------------------------
\section{Preprocessing \& Parameters}
% ----------------------------------------------------------------------------
Following our meeting, I decided to re-run the upstream processing of the Quant-Seq samples using the latest build of the \textit{Danio rerio} genome (GRCz12tu; Apr 4, 2025). I tried to keep the analysis consistent with what Maria did previously, running a UMI-sensitive pipeline with the following command:

\begin{lstlisting}[
    basicstyle=\footnotesize\ttfamily,
    breaklines=true,
    breakatwhitespace=true,
    columns=flexible,
    keepspaces=true
]
cd $SCRATCH/lena_quantseq
sbatch submit_nextflow_clip.sh \
  nf-core/rnaseq \
  --input samplesheet.csv \
  --outdir results/ \
  --fasta reference/genome.fna \
  --gtf reference/genome.gtf \
  --with_umi \
  --umitools_extract_method regex \
  --umitools_bc_pattern \
    "'^(?P<cell_1>.{6})(?P<umi_1>.{4}).*'" \
  --featurecounts_group_type gene_id \
  -c nextflow.config \
  -with-tower
\end{lstlisting}

The raw output files for the pipeline can be found in \verb|/bioinfo/andres.gordo/rnaseq_nfcore_out_umi.tar.gz|. The main output file that was used for downstream analysis is the gene counts matrix \verb|salmon.merged.gene_counts.tsv|. Normalisation and differential expression analysis was performed using DESeq2 in R (see code for details), with log2 fold-change shrinkage applied via a zero-centred Normal prior (\texttt{lfcShrink}, \texttt{type = "normal"}) to stabilise extreme effect-size estimates. From now onwards, a differentially expressed gene (DEG) is defined as one with an adjusted p-value $< 0.05$ (BH) and an absolute log2 fold change $\geq 1.0$. Genes with less than 10 counts across at least 3 samples were filtered out prior to DE analysis. This, as will be observed later, resulted in wnt11---member of the Nodal score---being lost.

\begin{figure}[H]
\centering
\includegraphics[width=\columnwidth]{results/compare_annotations/annotation_comparison_figure.pdf}
\caption{Annotations differences between the GRCz11 and GRCz12tu builds, showing the number of genes that were added, removed, or kept in each biotype category. None of the DEGs obtained in the following analyses were affected by these changes.}
\label{fig:annotation_changes}
\end{figure}

I also quickly compared the differences between the GRCz11 and GRCz12tu builds, and found that the biggest difference lies in an increase in the number of lncRNAs, which are not relevant for this analysis. I also observed some protein-coding additions and removals (mostly computational), although none of the DEGs obtained in the following analyses were affected by those changes (Fig~\ref{fig:annotation_changes}).

\section{Maria's Re-Analysis}

I started by quickly re-exploring one of Maria's previous analyses to 1) ensure that I could reproduce her results and 2) investigate if any newly-annotated gene popped up as DEG. Figure~\ref{fig:single} shows a result from Experiment 1, whereby the exposure time, but not the concentration, of Activin led to significant changes in the expression of genes from the Nodal score.

% --- Single column figure ---
\begin{figure}[H]
\centering
\includegraphics[width=\columnwidth]{results/nodal_heatmap_exp1_averaged.pdf}
\caption{Z Score of counts from the Nodal score genes across different time points and Activin concentration in Experiment 1.}
\label{fig:single}
\end{figure}




\begin{figure}[H]
\centering
\includegraphics[width=\columnwidth]{results/pca_combined.pdf}
\caption{PCA of gene expression across different time points and Activin concentration in Experiment 1 and Experiment 2.}
\label{fig:pca_combined}
\end{figure}

\section{Consistency of Activin Response}

\begin{figure}[H]
\centering
\includegraphics[width=\columnwidth]{results/q1_combined_volcano.pdf}
\caption{Volcano plots of DEGs in each experiment. In purple the shared genes. DEGs defined as adjusted p-value $<= 0.05$ and absolute log2 fold change $>= 1.0$.}
\label{fig:exp1_vs_exp2}
\end{figure}

The first thing on the list was to compare the DGE expression for $0$~ng/ml Activin vs.\ $15$~ng/ml Activin in Experiment~1 vs.\ Experiment~2. As can be observed in Figure~\ref{fig:exp1_vs_exp2}, there is good consistency between both experiments, with most genes following the same trend (up- or down-regulated). Experiment~1 yielded $1{,}311$ DEGs, while Experiment~2 had $1{,}076$ DEGs, with a total of $845$ DEGs in common. A quick look at one of the top genes, id3, reveals it as a DNA remodeller that is expressed at the dorsal side of the embryo. The list of all and shared DEGs can be found in q1\_exp1\_0vs15\_activin\_240min.csv, q1\_exp2\_0vs15\_dmso.csv, and q1\_shared\_de\_genes.csv, respectively.

\section{Nodal Score Exploration}

Next on the list was to re-explore how the selected list of genes belonging to the Nodal score varied in expression after 240 minutes of constant Activin exposure or Activin inhibition by SB50 at different timepoints. Figure~\ref{fig:nodalscore} shows the cumulative read counts of those Nodal score genes depending on the treatment. The `clean' treatment with Activin reaches the highest stimulation, whereas treatment with SB50 dampens the response. As was observed before, the longer the initial exposure to Activin, the more irreversibly committed the cells are, thereby responding less to the SB50 treatment. The statistical test resembles the one performed for GSEA (see code for details).

\begin{figure}[H]
\centering
\includegraphics[width=\columnwidth]{results/q2_nodal_score_cumulative.pdf}
\caption{Nodal score cumulative expression across treatments. A Kruskal-Wallis test was performed to assess significance across all conditions, followed by pairwise Wilcoxon tests with BH correction. Asterisks indicate significant pairwise comparisons against the control (0ngml DMSO): * p $< 0.05$, ** p $< 0.01$, *** p $< 0.001$.}
\label{fig:nodalscore}
\end{figure}


\section{Blocking Effect of SB50}

Next, I wanted to explore how many of the genes that were altered by Activin were effectively blocked by the SB50 treatment. To do so, I first obtained the list of genes that were significantly upregulated by Activin (15ng/ml DMSO vs.\ 0ng/ml DMSO), and then checked how many of those were not significantly differentially expressed when comparing 15ng/ml Activin + SB50 vs.\ 0ng/ml DMSO. All samples were from Experiment~2. I found a total of 826 DEGs comparing Activin vs.\ the baseline control in Experiment~2. Of these, 533 were upregulated and 293 were downregulated. On the other hand, 227, 609, and 887 DEGs were found after inhibiting Activin with SB50 at 60min, 120min, and 180min, respectively. Again, this is consistent with the idea that longer exposure times to Activin lead to more irreversible cell fates. Figure~\ref{fig:sb50_blocking} aims at answering whether genes induced by Activin are blocked by SB50. Indeed, the highest proportion of genes being blocked (in blue, significant change in Activin vs.\ control, but not significant in Activin + SB50 vs.\ control) is observed at the earliest timepoint (60min), whereas the lowest proportion of blocked genes is observed at the latest timepoint (180min). Importantly, no gene has a reversed response (i.e., an upregulation by Activin coincides with a downregulation by SB50, or vice versa): either they are induced by Activin and blocked by SB50, induced by Activin and not blocked by SB50, or induced by SB50. Another way of interpreting the figure is that the closer the distribution is to a perfect diagonal line (equal proportions of blocked and not blocked genes), the less effective the SB50 treatment is at blocking Activin-induced genes.

\begin{figure}[H]
\centering
\includegraphics[width=\columnwidth]{results/q3_analysis1_blocking.pdf}
\caption{Number and proportion of activin-induced genes that are blocked, not blocked, or reversed by SB50 treatment at different timepoints.}
\label{fig:sb50_blocking}
\end{figure}


\section{SB50 Effect vs.\ Timed Exp 1}

A similar yet different analysis from the previous is the question of comparing the effect of SB50 treatment with the corresponding matching times of collection in Experiment~1. Figure~\ref{fig:sb50_vs_time} shows this approach. The way I interpret it is manifold. First, looking at the X axis, which compares the effect of SB50 at endpoint with the effect of Activin at endpoint (both from Experiment~2), we again observe how SB50 is more effective at blocking---downregulating---genes when added at the earliest point (orange points more abundant on the left side; they decrease in number as we increase the treatment time). Second, looking at the Y axis, we are comparing instead the effect at endpoint of SB50 (Experiment~2) with the state of cells that were exposed to Activin at a given timepoint. Therefore, the gene expression differences between these two conditions reflect two different temporal states. In the first plot (top left), we compare adding SB50 at 60min and collecting at 240 minutes with Activin-induced cells collected at 60 minutes. As expected, many genes are differentially expressed (mostly upregulated) between 60 minutes and 240 minutes, regardless of the SB50 treatment. However, as we delay not only the time of SB50 addition but also of Activin-induced cells collection, the two conditions temporally converge, and the number of DEGs decreases.

These two figures suggest that:
\begin{enumerate}
  \item SB50 is more effective at blocking Activin-induced genes the earlier it is added, hinting at a cell commitment process that is time-dependent.
  \item Uncoupling temporal states from Activin inductions allows us to dissect which genes are Activin-dependent vs.\ time-dependent.
\end{enumerate}

\begin{figure}[H]
\centering
\includegraphics[width=\columnwidth]{results/q3_analysis2_comparisons.pdf}
\caption{Comparison of gene expression changes induced by SB50 treatment at endpoint vs.\ activin-induced changes at different timepoints. Each panel corresponds to a different SB50 addition time (60min, 120min, 180min).}
\label{fig:sb50_vs_time}
\end{figure}





\section{Gene Candidates \& ChIP-Seq}

Finally, I explored the dynamics of gene expression of the list of candidates genes alongside the ones encompasing the nodal Score, and integrated insights from the publsihed 2014 ChIP-Seq dataset of Smad2 and Eomesa. To do that, I first asked whether these genes appeared in either ChIP-seq dataset. Figure~\ref{fig:chip} shows that information, including if the FoxH1 motif was also found around the peak.


\begin{figure}[t]
\centering
\includegraphics[width=\columnwidth]{results/q4_chipseq_binding_heatmap.pdf}
\caption{Heatmap showing if the candidate or nodal score genes were found in the ChIP-seq experiment, including the presence of the FoxH1 motif around the peak.}
\label{fig:chip}
\end{figure}

Next, I analysed the ppaterns of gene expression in respone to activin (Exp 1) of these candiadte genes and the Nodal score. I found that, except sna1b--which is downregulated--, all genes are either actiavted consistently, or activated at early timepoints and then downregulated at later timepoints. Thisis shown in Figure~\ref{fig:temporal_clusters}.

\begin{figure*}[t]
\centering
\includegraphics[width=\textwidth]{results/q4_temporal_clusters.pdf}
\caption{Heatmap showing the temporal patterns of gene expression in response to Activin (Experiment 1) for the candidate genes and the Nodal score.}
\label{fig:temporal_clusters}
\end{figure*}

Consequently, Figure~\ref{fig:reversibility_profile} shows the same genes but how they respond to the Experiment 2 treatments.

\begin{figure}[t]
\centering
\includegraphics[width=\columnwidth]{results/q4_reversibility_profile.pdf}
\caption{Heatmap showing the temporal patterns of gene expression in response to Activin (Experiment 2) for the candidate genes and the Nodal score.}
\label{fig:reversibility_profile}
\end{figure}

And finally, Figure~\ref{fig:integrated_overview} shows an integrated overview of gene expression patterns for the candidate genes and the Nodal score across both experiments.


\begin{figure}[t]
\centering
\includegraphics[width=\columnwidth]{results/q4_integrated_overview.pdf}
\caption{Heatmap showing the integrated overview of gene expression patterns for the candidate genes and the Nodal score across both experiments.}
\label{fig:integrated_overview}
\end{figure}

\section{Molecular Function}

Next, I performed a quick Gene Ontology enrichment analysis using the clusterProfiler package in R. I focused on the Molecular Function ontology and explored the top enriched terms for the different comparisons performed so far. Figure~\ref{fig:go_mf} shows a selection of interesting terms that were enriched in different comparisons. The Exp~1 figure shows the Gene Ontology terms of the DEGs of each Activin concentration at each timepoint compared to the non-induced control samples. It shows predominancy of terms related to growth, signalling pathways, and transcription. Interestingly, the terms related to signalling receptors appear in every condition except for the latest timepoint, regardless of concentration, suggesting a negative feedback loop whereby Activin at first prompts the transcription of signalling receptors, but later on these get downregulated again.



\begin{figure*}[t]
\centering
\includegraphics[width=\textwidth]{results/go_mf_combined_publication.pdf}
\caption{Molecular Function GO term enrichment analysis for Experiment 1 (top) and Experiment 2 (bottom). In Experiment 1, each panel corresponds to a different timepoint (60min, 120min, 240min), with different Activin concentrations (0ng/ml, 5ng/ml, 15ng/ml) compared to the control (0ng/ml). In Experiment 2, each panel corresponds to a different comparison: Activin vs.\ Control (left), SB50 vs.\ Control (centre), SB50 vs.\ Activin (right). Only the top 10 enriched terms per condition are shown.}
\label{fig:go_mf}
\end{figure*}


\section{Conclusion}

So far I think this is one of the best bulk rnaseq datasets I have ever worked with in terms of quality: showing a high consistency in the transcriptomic response to Activin, specifically noting that exposure time is a more significant driver of gene expression changes than concentration. The analysis of the Nodal score and SB50 inhibitor treatments suggests a time-dependent cell commitment process: early inhibition effectively blocks Activin-induced genes, but later inhibition fails to reverse the phenotype as cells become "committed" to their fate. Finally, Gene Ontology enrichment points toward a negative feedback loop in signaling receptors, which are initially upregulated by Activin but subsequently downregulated, a trend that SB50 can only partially rescue depending on the timing of the intervention.

% --- Full width figure (spans both columns) ---
% \begin{figure*}[t]
%     \centering
%     \includegraphics[width=\textwidth]{path/to/image.pdf}
%     \caption{Caption for full-width figure spanning both columns.}
%     \label{fig:fullwidth}
% \end{figure*}

% --- Two subfigures side by side (full width) ---
% \begin{figure*}[t]
%     \centering
%     \begin{subfigure}[t]{0.48\textwidth}
%         \centering
%         \includegraphics[width=\textwidth]{path/to/imageA.pdf}
%         \caption{Panel A}
%         \label{fig:panelA}
%     \end{subfigure}
%     \hfill
%     \begin{subfigure}[t]{0.48\textwidth}
%         \centering
%         \includegraphics[width=\textwidth]{path/to/imageB.pdf}
%         \caption{Panel B}
%         \label{fig:panelB}
%     \end{subfigure}
%     \caption{Overall caption for both panels.}
%     \label{fig:panels}
% \end{figure*}

% ============================================================================

\end{document}
